% ---------------------------- Problem 1----------------------------------
\subsubsection*{\center Задача № 1.}
{\bf Условие.~}
Разложить в ряд Фурье заданную функцию $f(x)$, построить графики $f(x)$ и суммы ее ряда Фурье. Если не указывается, какой вид разложения в ряд необходимо представить, то требуетчя разложить функцию либо в общий тригонометрический ряд Фурье, либо следует выбрать оптимальный вид разложения в зависимости от данной функции.

$$
f(x)= \cos{(4x)}, \quad 0 < x < \pi/4, \quad\text{использовать разложение по синусам кратных дуг.}
$$
{\bf Решение.~}	
%График
\begin{center}
	\begin{tikzpicture}
	\begin{axis}[xmin=-1,	xmax=3.5, 	ymin=-1,	ymax=0.5,
	width=0.5\textwidth,
	height=0.4\textwidth,
	axis x line=middle,
	axis y line=middle,
	every axis x label/.style={at={(current axis.right of origin)},anchor=west},
	every inner x axis line/.append style={|-latex'},
	every inner y axis line/.append style={|-latex'},
	minor tick num=1,			
	axis equal=true,
	xlabel=$x$,
	ylabel=$y$,
	samples=100,
	clip=true,
	]
	
	\addplot[color=black, line width=1.5pt,domain=0:0.7854]{cos(deg(4*x))};
	\end{axis}
	\end{tikzpicture}
\end{center}
\noindent
Построим тригонометрический ряд Фурье вида
$$
f(x)=\sum_{n=1}^\infty
	\left(b_n \sin\frac{\pi n x}{l}\right),\quad\text{где}\,l=\frac{\pi}{4}.
$$
\noindent
Вычислим коэффициент
$$
\begin{array}{rcl}
b_n &=& \displaystyle\frac{8}{\pi}
	\int\limits_0^\frac {\pi}{4}
	\cos4x \sin(4nx)\,dx =
 \displaystyle\frac{1}{\pi}\left(
	\left.\frac{\cos (4x(n+1))}{n+1} + \frac{\cos (4x(n-1))}{n-1}\right) \right|_\frac{\pi}{4}^0 =	\\[12pt]
	&=& \displaystyle \frac{1}{\pi} \left(1-(-1)^{n+1}\right) \left(\frac{1}{n+1} + \frac{1}{n-1}\right).
\end{array}
$$

Видно, что при $n=1$ знаменатели в $b_n$ обращаются в нуль. Для нахождения первого элемента ряда Фурье вычислим предел (с учётом того, что при целых n
$ -(-1)^{n+1} = -\cos(\pi n + \pi) = -\cos(\pi n - \pi) = \cos(\pi n)$):


$$
b_1 = \displaystyle \lim_{n\to1} \frac{1}{\pi} \left(1 + \cos(\pi n) \right) \frac{2n}{n^{2}-1} =
\lim_{t\to0} \frac{2t(1-\cos(\pi t))}{\pi  t^{2}} = \lim_{t\to0} \frac{\pi^{2}  t^{2}}{\pi t} = 0
$$

Таким образом, $b_1$=0, следовательно, первый элемент ряда также равен нулю.
\\[12pt]
\\[12pt]
\\[12pt]
Применив теорему Дирихле о поточечной сходимости ряда Фурье, видим, что построенный ряд Фурье сходится
к периодическому (с периодом $T=\frac{\pi}{2}$) продолжению исходной функции, дополненной до нечётной, при всех $x\ne \frac{\pi}{4}k$, и
$S(\frac{\pi}{4}k)=0$ при $k=0,\pm1,\pm2,\ldots$, где $S(x)$ --- сумма ряда Ферье.
График функции $S(x)$ имеет следующий вид
\begin{center}
	\begin{tikzpicture}
	\begin{axis}[xmin=-pi, xmax= pi, ymin=-1, ymax=0.5,
	width=0.8\textwidth,
	height=0.4\textwidth,
	axis x line=middle,
	axis y line=middle,
    ytick={-1, -0.5 , 0, 0.5,...,1},
    xtick={-0.75*pi, -pi/2, -pi/4,0,pi/4,pi/2, 0.75*pi},
	every axis x label/.style={at={(current axis.right of origin)},anchor=west},
	every inner x axis line/.append style={|-latex'},
	every inner y axis line/.append style={|-latex'},
	%minor tick num= 0.125*pi,			
	axis equal=true,
	xlabel=$x$,
	ylabel=$S(x)$,
	samples=100,
	clip=true,
	]
	\addplot[color=black, line width=1.5pt,domain=0:pi/4]{cos(deg(4*x))};
    \addplot[color=black, line width=1.5pt,domain=-pi/4:0]{-cos(deg(4*x))};
    \addplot[color=black, line width=1.5pt,domain=pi/4:pi/2]{-cos(deg(4*x))};
    \addplot[color=black, line width=1.5pt,domain=-pi/2:-pi/4]{cos(deg(4*x))};
    \addplot[color=black, line width=1.5pt,domain= pi/2:0.75*pi]{cos(deg(4*x))};
    \addplot[color=black, line width=1.5pt,domain=-pi/2:-0.75*pi]{-cos(deg(4*x))};


	\addplot[
	mark=*,
	mark options={fill=black, draw=black},
	only marks,
	] coordinates {(0, 0)  (0.7854, 0) (-0.7854, 0) (-1.5708, 0) (1.5708, 0) (2.3562, 0) (-2.3562, 0)};
	\end{axis}
	\end{tikzpicture}
\end{center}
\noindent
\textbf{Ответ:}
\[
\begin{split}
&f(x)= \sum_{n=2}^\infty\frac{1}{\pi} \left(1-(-1)^{n+1}\right) \left(\frac{1}{n+1} + \frac{1}{n-1}\right) \sin(4nx), x\ne 3k; \\
&S(3k)= 0, \text{ при } k\in\mathbb{Z}.
\end{split}
\]




% ---------------------------- Problem 2----------------------------------
\subsubsection*{\center Задача № 2.}
{\bf Условие.~}
Для заданной графически функции $y(x)$ построить ряд Фурье в комплексной форме, изобразить график суммы построенного ряда

%График
\begin{center}
	\begin{tikzpicture}
	\begin{axis}[xmin=-1,	xmax=5.4, 	ymin=-1,	ymax=1.45,
	width=0.5\textwidth,
	height=0.4\textwidth,
	axis x line=middle,
	axis y line=middle,
	every axis x label/.style={at={(current axis.right of origin)},anchor=west},
	every inner x axis line/.append style={|-latex'},
	every inner y axis line/.append style={|-latex'},
	minor tick num=1,			
	%axis equal=true,
	xlabel=$x$,
	ylabel=$y$,
	samples=100,
	clip=true,
	]
	\addplot[color=black, line width=1.5pt,domain=2:5] {-0.5};
	\addplot[color=black, line width=1.5pt,domain=0:2]{-0.75*x+1};
	\end{axis}
	\end{tikzpicture}
\end{center}

\noindent
\textbf{Решение.}\\

\noindent
Ряд Фурье в комплексной форме имеет следующий вид
\[
f(x) = \sum_{n=-\infty}^\infty c_n e^{i\omega nx},\quad c_n=\frac{1}{T}\int\limits_a^b f(x) e^{-i\omega nx}dx,\,\omega=\frac{2\pi}{T}.
\]
В нашем примере $ a=0,b=5,T=5,\omega=2\pi/5$,
найдем коэффицинеты $c_n,\,n=0,\pm1,\pm2,\ldots$
где $\omega=2\pi/T,\,T=5.$
$$
\begin{array}{rcl}
%c_0 &=&\displaystyle\frac{1}{3} \int\limits_0^3 f(x)dx=\frac{a_0}{2}=-\frac{5}{6},\\[12pt]
c_n &=&\displaystyle\frac{1}{5}\left(
\int\limits_0^2
(-\frac{3}{4}x+1)e^{-i\omega nx}dx- \int\limits_2^5
\frac{1}{2}e^{-i\omega nx}dx \right) ={}\\[12pt]
&=&\displaystyle\frac{1}{5}\left(
\left[ \frac{i}{\omega n} e^{-i\omega nx}\left(-\frac{3}{4}x+1+\frac{3i}{4\omega n}\right)\right] \bigg|_0^2
-\left.\frac{i}{2\omega n} e^{-i\omega nx}\right|_2^5\right) = \\[12pt]
&=&\displaystyle\frac{25}{8\pi^{2} n^{2}} \left( 3\sin^{2} \left(\frac{2\pi}{5} n  \right) + i \frac{ 15\sin \left(\frac{4\pi}{5} n  \right) - 12\pi n}{10}  \right).
\end{array}
$$
\noindent
Применив теорему Дирихле о поточечной сходимости ряда Фурье, видим, что построенный ряд Фурье сходится
к периодическому (с периодом $T=5$) продолжению исходной функции при всех $x\ne 5k$, и $S(5k)=0.25$ при
$k=0,\pm1,\pm2,\ldots$, где $S(x)$ --- сумма ряда Фурье. График функции $S(x)$ имеет вид
\begin{center}
	\begin{tikzpicture}
	\begin{axis}[xmin=-10, xmax=10.5, ymin=-2, ymax=1.75,
	width=0.8\textwidth,
	height=0.4\textwidth,
	axis x line=middle,
	axis y line=middle,
	every axis x label/.style={at={(current axis.right of origin)},anchor=west},
	every inner x axis line/.append style={|-latex'},
	every inner y axis line/.append style={|-latex'},
	minor tick num=1,			
	%axis equal=true,
	xlabel=$x$,
	ylabel=$S(x)$,
	samples=100,
	clip=true,
	]
	\addplot[color=black, line width=1.5pt,domain=2:5] {-0.5};
	\addplot[color=black, line width=1.5pt,domain=0:2]{-0.75*x+1};
	\addplot[color=black, line width=1.5pt,domain=7:10] {-0.5};
	\addplot[color=black, line width=1.5pt,domain=5:7]{-0.75*x+4.75};
	\addplot[color=black, line width=1.5pt,domain=-3:0] {-0.5};
	\addplot[color=black, line width=1.5pt,domain=-5:-3]{-0.75*x-2.75};
    \addplot[color=black, line width=1.5pt,domain=-8:-5] {-0.5};
	\addplot[color=black, line width=1.5pt,domain=-10:-8]{-0.75*x-6.5};
	\addplot[
	mark=*,
	mark options={fill=black, draw=black},
	only marks,
	] coordinates {(5, 0.25) (10, 0.25) (0, 0.25) (-5, 0.25) (-10, 0.25)};
	\end{axis}
	\end{tikzpicture}
\end{center}

\noindent
\textbf{Ответ:}
\[
\begin{split}
&f(x)=\sum_{n=-\infty}^\infty\frac{25}{8\pi^{2} n^{2}} \left( 3\sin^{2} \left(\frac{2\pi}{5} n  \right) + i \frac{ 15\sin \left(\frac{4\pi}{5} n  \right) - 12\pi n}{10}  \right) e^{\tfrac{i2\pi nx}{5}},~ x\ne 5k; \\
&S(5k)=\frac{1}{4},\quad\text{при}~k\in\mathbb{Z}.
\end{split}
\]


% ---------------------------- Problem 3----------------------------------
\subsubsection*{\center Задача № 3.}
{\bf Условие.~}\\
Найти резольвенту для интегрального уравнения Вольтерры со следующим ядром
$$K(x,t)=te^\frac{x^2-t^2}{2}.$$

\noindent
{\bf Решение.~}\\
\noindent

Из рекурентных соотношений получаем
$$
\begin{array}{rcl}
K_1(x,t)&=&\displaystyle te^\frac{x^2-t^2}{2}, \\[12pt]
K_2(x,t)&=&\displaystyle\int\limits_t^x K(x,s)K_1(s,t)ds = \int\limits_t^x se^\frac{x^2-s^2}{2} te^\frac{s^2-t^2}{2} ds = \frac {te^\frac{x^2-t^2}{2}}{2}\cdot(x^{2}-t^{2}),\\[12pt]
K_3(x,t)&=&\displaystyle\int\limits_t^x K(x,s)K_2(s,t)ds = \int\limits_t^x se^\frac{x^2-s^2}{2} \frac{te^\frac{s^2-t^2}{2}}{2} (s^2-t^2) ds = \frac {te^\frac{x^2-t^2}{2}}{4}\cdot\left(\frac {x^{4}-t^{4}}{2} - t^2(x^{2}-t^{2})\right),\\[12pt]
K_4(x,t)&=&\displaystyle\int\limits_t^x K(x,s)K_3(s,t)ds = \int\limits_t^x se^\frac{x^2-s^2}{2} \frac{te^\frac{s^2-t^2}{2}}{4} \left(\frac{s^4-t^4}{2} - t^2(s^2-t^2)\right) ds = \\[12pt]
&=&\displaystyle \frac {te^\frac{x^2-t^2}{2}}{8}\cdot\left(\frac{x^6-t^6}{6}-t^2\frac {x^{4}-t^{4}}{2} + t^4(x^{2}-t^{2})\right).\\[12pt]
K_j(x,t)&=&\displaystyle\frac {te^\frac{x^2-t^2}{2}}{2^{j-1}}\cdot \sum_{k=1}^{j-1} (-1)^{j-1-k}\frac{x^{2k}-t^{2k}}{k!} t^{2^{j-1-k}}      ,\quad j=\mathbb{N},\quad j>1.
\end{array}
$$
Подставляя это выражение для итерированных ядер, найдем резольвенту
$$
R(x,t,\lambda)=te^\frac{x^2-t^2}{2}\left(1+\sum_{j=2}^\infty \frac {\lambda^{j-1}}{2^{j-1}}\cdot \sum_{k=1}^{j-1} (-1)^{j-1-k}\frac{x^{2k}-t^{2k}}{k!} t^{2^{j-1-k}}\right),
\quad j=2,3,\ldots
$$

